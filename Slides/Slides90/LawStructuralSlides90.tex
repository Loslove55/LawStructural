\documentclass{beamer}

%\usepackage{beamerthemeshadow} %// Activate for custom appearance
%\usetheme{Warsaw}
\usetheme{Madrid}
\setbeamertemplate{headline}{}
\setbeamertemplate{footline}[frame number]
\definecolor{uiGold}{RGB}{255,225,0}
\usecolortheme[named=uiGold]{structure}
\setbeamercolor{title}{fg=black}
\setbeamercolor{frametitle}{fg=black}
\setbeamercolor{blocktitle}{fg=black}
\setbeamercolor{caption name}{fg=black}
\setbeamercolor{button}{fg=black}
\setbeamercolor{item projected}{fg=black}
\setbeamertemplate{theorems}[numbered]

\usepackage{multicol}
\usepackage{graphicx}
\usepackage{textpos}
\usepackage{epstopdf}
\usepackage{amsmath}
\usepackage{bbm}

% To add logo to upper right corner of slides.
%\addtobeamertemplate{frametitle}{}{
%\begin{textblock*}{90mm}(.95\textwidth,-.9cm)
%\includegraphics[height=.8cm,width=.8cm]{../dome.jpg}
%\end{textblock*}}


\title{Information Asymmetry in Profit-generating Graduate Education Markets: A Structural Approach to Law Schools}
\author{Philip J. Erickson}
\institute{University of Iowa}
\date{\today}


%\newtheorem{lemma}[theorem]{Lemma}
\newtheorem{proposition}[theorem]{Proposition}
%\newtheorem{corollary}[theorem]{Corollary}

\theoremstyle{definition}
\newtheorem{assumption}{Assumption}[section]
\newtheorem{remark}{Remark}[section]
%\useoutertheme{infolines} 

% Source:
%   http://tex.stackexchange.com/questions/34921/
%   how-to-overlap-images-in-a-beamer-slide
\def\Put(#1,#2)#3{\leavevmode\makebox(0,0){\put(#1,#2){#3}}}

\newcommand{\E}{\mathbb{E}}
\newcommand{\ind}{\mathbbm{1}}

\begin{document}

{
\setbeamertemplate{footline}{} 
\begin{frame}[noframenumbering]
 \titlepage
\end{frame}
}

\frame{
  \frametitle{Introduction}
  Difficulties in market for training lawyers
  \begin{itemize}
    \item High (and increasing) tuition
    \item Few job openings (EMSI, BLS)
    \item Stable, but not growing, real wages
    \item More schools opening
  \end{itemize}
  Questions
  \begin{itemize}
    \item Too many law schools?
    \item Too many law students? Mismatch problem?
    % Too many: Enough that the expected returns from graduating from a given
    % school are less than the price of purchasing that education
    \item Tuition too high?
  \end{itemize}
  Approach
  \begin{itemize}
    \item Dynamic game (Ericson and Pakes ('95)), schools compete for students
    \item Asymmetric information between schools and students - use exogenous shock
  \end{itemize}
}

\frame{
  \frametitle{Contributions}
  \begin{itemize}
    \item Add to information asymmetry/market failure literature
    \begin{itemize}
      \item Asymmetric information leads to matching problem: wrong students pay wrong price for product (not worth expenditure for current students, crowding out potentials)
      \item Information helps consumers/producers self-correct
    \end{itemize}
    \item Market-specific empirical results
    \begin{itemize}
      \item Schools are affected differentially by information change
      \item Producer (Consumer) surplus negatively (positively) affected
      \item Total welfare increases by \$363 million
    \end{itemize}
    \item Application of EP model to education market
    \item Integration of asymmetric information effect with EP
  \end{itemize}
}

\frame{
  \frametitle{Literature}
  \begin{itemize}
    \item {\bf Market-specific empirical results} [Rosen('92), Spur('87), Ehrenberg('88), Oyer and Schaefer('10, '10), MacIntyre and Simkovic('13)]
    \item {\bf Application of Ericson Pakes ('95) framework to higher education} [Hickman('09), Azevedo('12), Fu('14), Fillmore('15)]
    \item {\bf Integration of asymmetric information effect with EP}
    \item {\bf Add to information asymmetry/market failure literature} [Manski('04), Arcidiacono, Hotz, Kang('11), Dranove, Kessler, McClellan, Satterthwaite('03), Jin and Sorenson('06)]
  \end{itemize}
  
}


\frame{
  \frametitle{Information Structure}
  \begin{itemize}
    \item Information governed by ABA, published in ``ABA-LSAC Official Guide''
    \item Pre-2010 primary ROI statistics:
    \begin{itemize}
      \item \% Placed in a job 9-months after graduation
      \item \% In Law, Business, Government, public interest, clerk, academia
    \end{itemize}
    \item Wages guessed from BLS or NALP - not school conditional and low reporting
  \end{itemize}
}

\frame{
  \frametitle{Information Structure}
  \begin{itemize}
    \item Post-2010 ROI statistics
    \begin{itemize}
      \item 144 placement types
      \item Example: \# Employed in a law firm with 11-25 employees, full-time, short-term
    \end{itemize}
    \item Informative:
  \end{itemize}
  \begin{table}[htdp]
  \begin{center}
  \begin{tabular}{lcccccc}
  \hline
  \hline
  Firm Size & 2-10 & 11-25 & 26-50 & 51-100 & 101-250 & 251 or More \\
  Salary & 73,000 & 73,000 & 86,00 & 91,000 & 110,000 & 130,000 \\
  \hline
  \end{tabular}
  \end{center}
  \caption{Median Lawyer Starting Salary per Firm Size (2011)}
  \label{tab:nalp}
  \end{table}
  
}

\frame{
  \frametitle{Data}
  \label{fr:data}
  School-level: ABA-LSAC Report and US News and World Report
  \begin{itemize}
    \item Unit of analysis: one school per year
    \item Academic years beginning 1998-2013
    \item Variables
    \begin{itemize}
      \item Quality of program: US News rank, placement, student/faculty ratio, Bar passage rate
      \item Quality of students: undergrad GPA, LSAT
      \item Revenue: class size, tuition
      \item Cost to students: median grant, percent grants, room/board expenses, cost of books
      \item Selectivity: accepted, acceptance rate
      \item Private-info ROI: \%25/\%75 private sector starting salary, \% reporting
    \end{itemize}
  \end{itemize}
  Student-level: LawSchoolNumbers.com
  \begin{itemize}
    \item Unit of analysis: possible student-school match
    \item Includes applicant profile, name of school, match outcome
  \end{itemize}
}

\frame{
  \frametitle{Information Effect}
  \begin{itemize}
    \item Based off reporting change in 2010
    \item Expected: Schools with new lower reports should have to adjust combination of tuition, quality thresholds, class size
    \item How information is internalized:
    \begin{itemize}
      \item Direct report
      \item US News Rank (information aggregation)
    \end{itemize}
    \item Both highly correlated
  \end{itemize}
}

\frame{
  \frametitle{Evidence of Information Effects}
  \begin{figure}[htbp]
    \begin{center}
      \includegraphics[width=.6\textwidth]
                      {../../Results/Initial/Figures/OverallRankAvecompact.pdf}
    \end{center}
  \end{figure}
  
  \Put(-75, 250){
    \includegraphics[width=.6\textwidth]
    {../../Results/Initial/Figures/OverallRankAveLegendcompact.pdf}
  }
}

\frame{
  \frametitle{Evidence of Information Effects}
  \[
  	y_{it} = \beta_0 + \beta_1 Rank_{it} + \beta_2 Post2010_t +
  			     \beta_3 (Rank_{it} * Post2010_t) + \gamma X_{it} + \varepsilon_{it}
  \]
  
  \begin{figure}[htbp]
    \begin{center}
      \includegraphics[width=.5\textwidth]
                      {../../Results/FirstStage/Figures/rankDinDcompact.pdf}
    \end{center}
  \end{figure}
  \hfill\hyperlink{fr:ratio}{\beamerbutton{Ratio}}
}

\frame{
  \frametitle{Reduced Form Takeaways}
  \begin{itemize}
    \item Tuition and GPA more elastic than class size and LSAT
    \item Consistent with inelastic supply wrt. demand
    \item GPA vs. LSAT also consistent with high rank premium
    \item New information gives some cardinality to rank
  \end{itemize}
}

\frame{
  \frametitle{Structural Model}
  \label{fr:structural_model}
  \begin{itemize}
    \item Why do we need structural model?
    \begin{itemize}
      \item Welfare - rank premium and student problem
      \item Entry/Exit
    \end{itemize}
    \item Dynamic endogenous capacity game (Ericson and Pakes ('95), Doraszelski and Satterthwaite ('10))
    \item Schools compete with others for student enrollment (Fu ('04))
    \begin{itemize}
      \item Application, Admission, Enrollment
      \item Embed subgame in Ericson and Pakes framework
    \end{itemize}
    \item Includes entry/exit decisions
  \end{itemize}
}

\frame{
  \frametitle{Players}
  Schools
  \begin{itemize}
    \item Differentiated by rank $R_j$ (zero indicates non-participation)
    \item State vector $(R, g)$, $R_j\in R$
    \item Demand growth term $g$
    \item $\overline{N}$ possible schools, $J$ participants
    \item Fixed capacity $\overline{Q}_j$, $\sum_{j=1}^{\overline{N}}\overline{Q}_j < 1$
  \end{itemize}
  Students
  \begin{itemize}
    \item Unit mass
    \item Ability endowment $A_i$
    \item Observable signal of ability $(LSAT_i, GPA_i)$
    \item $A\sim N(f_A(LSAT, GPA), \sigma_A^2)$
  \end{itemize}
}

\frame{
  \frametitle{Stage Game: Timing}
  \begin{enumerate}
    \item School announces tuition (Bertrand-type competition)
    \item Students make application decisions, schools choose admission policies (endogenous capacity models)
    \item Students learn results, make enrollment decisions
  \end{enumerate}
}

\frame{
  \frametitle{Stage Game: Payoffs}
  Student Payoffs
  \begin{itemize}
    \item School specific:
      \begin{align*}
        u_{ij} &= \overline{u}(A_i, R_j, I) + \epsilon_{ij} \\
        \epsilon_{ij}&\sim N(0, \sigma^2_u)
      \end{align*}
    \item Total: $U_{ij}(t) = u_{ij} - t_j$
  \end{itemize}
  School Payoffs
  \begin{itemize}
    \item Tuition revenue: $\tilde{\pi}_j = \int t_j dF_j(i)$
    \item Net donations: $D(R_j; \delta) = \delta_1 R_j + \delta_2 R_j^2$
    \item Entry/exit costs:
    \[
      \Phi(a_j; \kappa_j, \phi_j) =
      \begin{cases}
        -\kappa_j, & \text{ if the school is a new entrant} \\
        \phi_j, & \text{ if the school exits}
      \end{cases}
    \]
    \item Total profits: $\pi_j = \tilde{\pi}_j + D(r_j; \delta) + \Phi(a_j; \kappa_j, \phi_j)$
  \end{itemize}
}

\frame{
  \frametitle{Information}
  School private information
  \begin{itemize}
    \item School $j$ receives signal for applicant $i$: $\nu_{ji}\sim N(0, \sigma_\nu^2)$
    \item Expected ability: $E[A_i | LSAT_i, GPA_i, \nu_{ij}] = f_A(LSAT_i, GPA_i) + \nu_{ij}$
  \end{itemize}
  
  Student private information
  \begin{itemize}
    \item Latent type: $Ai$
    \item Preference shock: $\varepsilon_{ij}$
  \end{itemize}
}

\frame{
  \frametitle{Application, Admission, Enrollment}
  Define $X_i = (LSAT_i, GPA_i)$,  $S \equiv (t, I, (R, g))$, $\varepsilon_i \equiv (\varepsilon_{ij})_{j\in J}$
  
  Value to admitted student
  \begin{equation}
    w_i(O_i, A_i, \varepsilon_i | S) = \max\{0, \max_{j\in O_i} U_{ij}(t)\}
    \label{eq:val_students}
  \end{equation}
  Probability of $i$ being admitted to $j$
  \[
    p_j(A_i, X_i|S)
  \]
  Value of application portfolio $Y$
  \begin{equation}
    W(Y, A_i, X_i|S) \equiv \sum_{O\subseteq Y} \Pr(O|A_i, X_i, S)E[w(O, A_i, \varepsilon_i|S)] - C(|Y|)
  \end{equation}
  The probability that set $O$ of colleges admits student $i$
  \begin{equation}
    \Pr(O|A_i, X_i, S) = \prod_{j\in O}p_j(A_i, X_i|S)
                         \prod_{j'\in Y\backslash O} (1 - p_{j'}(A_i, X_i|S)).
  \end{equation}
  Student application problem
  \begin{equation}
    \max_{Y\subseteq\{1, \dots, J\}}\{W(Y, A_i, X_i|S)\}
  \end{equation}
}

\frame{
  \frametitle{Dynamics and Timing}
  Rank evolution
  \begin{equation}
    R_j' = f_R(R, x, \varepsilon_R; \psi) \in [1, \bar{R}]
    \label{eq:rank_evol}
  \end{equation}
  with $x = (x_j)_{j=1, \dots, \overline{N}}$ and $x_j \equiv (LSAT_{jM}, GPA_{jM}, A_j)$
  Dynamic timing:
  \begin{enumerate}
    \item Potential entrants draw fixed entry cost, make entry decision
    \item Incumbents draw scrap value, make exit decision
    \item Subsequent participants compete in application-admission game
    \item Investment matures, entry/exit occurs
    \item Rankings update
  \end{enumerate}
}

\frame{
  \frametitle{School Strategies and Value Functions}
  School strategies
  \begin{enumerate}
    \item $\sigma_{j1}: (R_j, \nu_{ij}, X_i | t_j, I) \rightarrow \{0, 1\}$
    \item $\sigma_{j2}: ((R_j, g), \xi_j|I) \rightarrow a_j$
  \end{enumerate}
  with $a_j$ tuition, entry/exit decisions, $\xi_j$ school's private information. Define $\sigma_j \equiv (\sigma_{j1}, \sigma_{j2})$
  
  Value Functions (Incumbent/Entrant)
  \begin{align}
    &V_{j}((R, g); \sigma_j, I, \theta) \label{eq:Vinc}\\
    &= \max\left\{\pi_j + \max\left\{\phi_j, \beta\int E_{\xi_j}V_j((R', g');\sigma_j, I, \theta, \xi_j)dP(R'; R, \sigma_j, I)\right\}\right\} \notag
  \end{align}
  and
  \begin{align}
    &V_{J}((R, G); \sigma_j, I, \theta) \label{eq:Vent}\\
    &= \max\left\{0, \beta\int E_{\xi_j}V_j((R', g');\sigma_j, I, \theta, \xi_j)dP(R'; R, \sigma_j, I) - \kappa_j\right\}. \notag
  \end{align}
  
}

\frame{
  \frametitle{Application-Admission Equilibrium}
  \begin{definition}
    Given tuition profile $t$, information regime $I$ and rank vector $(R, g)$, a symmetric, anonymous application-admission equilibrium, denoted $AE(S)$ is the vector $(d(\cdot|\cdot), Y(\cdot|\cdot), \sigma_{j1}(\cdot|\cdot), p(\cdot|\cdot))$ such that
    \begin{enumerate}
      \item $d(\cdot|\cdot)$ is an optimal enrollment decision
      \item Given $p(\cdot|\cdot)$, $Y(\cdot|\cdot)$ is an optimal college application portfolio
      \item For every $j$, given $(d(\cdot|\cdot), Y(\cdot|\cdot), p_{-j}(\cdot|\cdot))$, $\sigma^*_{j1}(\cdot|\cdot)$ is an optimal admissions policy, and $\sigma^*_{j1}(\cdot|\cdot) = \sigma^*_{j'1}(\cdot|\cdot)$ if $R_j = R_{j'}$
      \item $p_j(\cdot|\cdot) = \int\sigma^*_{j1}(\cdot|\cdot)\Phi(0, \sigma_\nu^2)$
    \end{enumerate}
  \end{definition}
}

\frame{
  \frametitle{Markov-perfect Equilibrium}
  \begin{definition}
    A symmetric, anonymous, Markov-perfect equilibrium for the market for training lawyers is the vector $(\sigma^*_j, d(\cdot|\cdot), Y(\cdot|\cdot), \sigma_{j1}(\cdot|\cdot), p(\cdot|\cdot))$ such that
    \begin{enumerate}
      \item For every $t$, $(d(\cdot|\cdot), Y(\cdot|\cdot), \sigma_{j1}(\cdot|\cdot), p(\cdot|\cdot))$ constitutes an AE(t)
      \item For every $j$, given $\sigma^*_{-j2}$, $\sigma^*_{j2}$ is optimal for college $j$ and $\sigma^*_{j2} = \sigma^*_{j'2}$ if $R_j = R_{j'}$
    \end{enumerate}
  \end{definition}
}

\frame{
  \frametitle{Empirical Strategy}
  Estimate entire game using Bajari, Benkard, Levin ('07) (BBL)
  \begin{itemize}
    \item Flexibly estimate first-stage policy functions, take as optimal
    \item Forward simulate value functions for optimal and perturbed policies
    \item Construct minimum-distance estimator based on equilibrium definition
  \end{itemize}
}

\frame{
	\frametitle{Identifying Assumptions}
		\begin{assumption}
		All schools play the same Markov-perfect equilibrium.
		\label{as:unique}
	\end{assumption}
    Restricting to symmetric, anonymous equilibria. Proof of existence from Doraszelski and Satterthwaite ('10).
	\begin{assumption}
	  Schools assume that information structure is permanent.
	\end{assumption}
  \begin{assumption}
    Let $\underline{R}$ be the least profitable ranking for a school. The mean and variance of the distribution of scrap values are such that
    \begin{enumerate}
      \item $\mu_\phi = \{\mu_\phi : \Pr(a=\text{exit}|R_j=\underline{R}, \text{No info}) < 0.01\}$
      \item $\sigma^2_\phi = \sigma^2_\kappa$
    \end{enumerate}
    \label{as:scrap}
  \end{assumption}
  Gives upper-bound identification
}

\frame{
  \frametitle{Application-Admission Game}
  Estimates for stage $k \in {1, 2, 3}$ and a student $i$ and school $j$ using gradient boosted classification trees:
  \begin{equation}
    f_k(LSAT_i, GPA_i, R_j, t_j, I_{\{0, 1\}}, year) \rightarrow 
    (0, 1)
    \label{eq:fs_appadmit_stages}
  \end{equation}
  \begin{itemize}
    \item Could use brute force approach (simulate market per period)
    \item Computationally intractable
    \item Instead, use flexible estimates in \eqref{eq:fs_appadmit_stages} to simulate outcome functions
      \[
        \tilde{f}(R_j, t_j, I_{\{0, 1\}}, year) \rightarrow
        ([120, 180], [0,1], \mathbb{R}^+)'
      \]
      for $LSAT$, $GPA$, and class size. 
    \item Estimate with boosted regression trees based on simulated outcomes
    \item Flavor of Hotz and Miller ('93) and 2SLS
  \end{itemize}
}

\frame{
  \frametitle{Tuition and State}
  Tuition
  \begin{itemize}
    \item Estimate with gradient boosted regression tree
    \item Use to fit the mapping
      \[
        f_T(R_j, I_{\{0, 1\}}, year) \rightarrow \mathbb{R}
      \]
  \end{itemize}
  Rank
  \begin{itemize}
    \item First, estimate using simple tobit model
      \[
        \tilde{R}'_{j} = f(\psi_0 + \psi_1 R_j + \psi_x x_j + \varepsilon_R)
      \]
    \item After standard expected value, normalize to discrete ranks (similar to US News process)
  \end{itemize}
}

\frame{
  \frametitle{BBL Estimator}
  \begin{align*}
  	V_j((R, g); \sigma, I, \theta) &= \mathbb{E}\left[\sum_{t=0}^{\infty}\beta^t\Psi_j(\sigma, (R_t, g_t), \varepsilon_t|(R_0, g_0) = (R, g))\right]\cdot \theta \\
    &= {\bf W}_j((R, g); \sigma)\cdot \tilde{\theta}
  \end{align*}
  \[
  	\Psi_j = [\tilde{\pi}_j, r_j, r_j^2]'
  \]
  \[
    \tilde{\theta}\equiv[1, \delta_1, \delta_2]
  \]
  \begin{equation}
    W_j((R, g); \sigma_j^*, \sigma_{-j}^*)\cdot\tilde{\theta} \geq
    W_j((R, g); \tilde{\sigma}_j, \sigma_{-j}^*)\cdot\tilde{\theta}
    \label{eq:eqbm_condition}
  \end{equation}
  \begin{equation}
    m(\tilde{\sigma}_j; \tilde{\theta}) = [
      W_j((R, g); \sigma_j^*, \sigma_{-j}^*) -
      W_j((R, g); \tilde{\sigma}_j, \sigma_{-j}^*)
    ]\cdot\tilde{\theta}.
  \end{equation}
  \begin{equation}
    \min_{\tilde{\theta}}Q_n(\tilde{\theta}) = \frac{1}{K}\sum_{k=1}^K
      1(m(\tilde{\sigma}_{j, k}; \tilde{\theta}) > 0)
      m(\tilde{\sigma}_{j, k}; \tilde{\theta})^2,
  \end{equation}
}

\frame{
  \frametitle{First-stage Estimates: Application-Admission Game}
  \label{fr:first_stage_app_admit}
  \begin{figure}[htbp]
    \begin{center}
      \includegraphics[width=0.7\textwidth]{../../Results/FirstStage/Figures/appadmit_importance.pdf}
      \caption{Application-Admission Game: Variable Importance}
      \label{fig:app_admit_importance}
    \end{center}  
  \end{figure}
  \hfill\hyperlink{fr:dynamics}{\beamerbutton{Dynamics}}
}

\frame{
  \frametitle{Second Stage: Conditional Producer Surplus}
  \label{fr:conditional_producer_surplus}
  \begin{figure}[htbp]
    \begin{center}
      \includegraphics[width=.5\textwidth]
                      {../../Results/SecondStage/Figures/value_diff.pdf}
    \end{center}
  \end{figure}
  \hfill\hyperlink{fr:val_funs}{\beamerbutton{Value Functions}}
}

\frame{
  \frametitle{Second Stage: Welfare Change}
  Producer Surplus
  \begin{itemize}
    \item Change calculated with $\Delta V(R) = V(R | info=1) - V(R | info=0)$
    \item Mean estimate:
      \[
        \Delta PS = \sum_{j=1}^{\overline{N}}\Delta V(\hat{R})_j \approx -\$212 \text{ million}
      \]
  \end{itemize}
  Consumer Surplus
  \begin{itemize}
    \item Value for $i$ of attending $j$ given by $U_{ij}(t)$
    \item Identification given by
      \[
        Pr(U_{ij}(t) > 0) = f_M(LSAT_i, GPA_i, Rank_j, Tuition_j) \equiv f_{Mij}
      \]
    \item Value of student $i$'s participation in market previously defined, denoted (with abuse of notation) $W_i(I=\{0, 1\})$
    \item Change in surplus for $i$: $\Delta W_i \equiv W_i(I=1) - W_i(I=0)$
    \item Mean estimate:
      \[
        \Delta CS = \int \Delta W_i dG(i) \approx \$575\text{ million}
      \]
  \end{itemize}
}

\frame{
  \frametitle{Total Surplus}
  \[
    \Delta TS = \Delta SS + \Delta PS = \$363\text{ million}
  \]
}

\frame{
  \frametitle{Conclusion}
  \begin{itemize}
    \item Not necessarily too many students being produced, but uninformed students were attending suboptimal programs and willing students were being driven out
    \item Policy can improve match, slow rate of entrance, and increase probability of school shut-down
    %\item Information does not change school profits or entry rates, but improve student welfare by (partially) mitigating mismatch problem
    \item General application: Information can strictly improve welfare by (partially) mitigating mismatch problem
  \end{itemize}
}


% -------------- %
%    APPENDIX    %
% -------------- %
\frame{
  \frametitle{Direct Report vs Aggregation}
  \label{fr:ratio}
  Direct Report
  \begin{itemize}
    \item Key indicator: ratio of full-information expected wages to partial-information expected wages:
    \[ Ratio_i = \sum \omega_i placement_i / \sum placement_i \]
    \item Weights $\omega_i$ determined by expected wages in placement type
    \item Transparency ratio highly persistent
    \item Impute over schools in time period for placement ``types''
  \end{itemize}
  Aggregation
  \begin{itemize}
    \item Included in rank calculation
    \item Used in conjunction with direct report
    \item Ranks now both ordinal and cardinal!
  \end{itemize}
}

\frame{
  \frametitle{Evidence of Information Effects: Ratio}
  \begin{figure}[htbp]
    \begin{center}
      \includegraphics[width=.6\textwidth]
                      {../../Results/Initial/Figures/RatioAvecompact.pdf}
    \end{center}
  \end{figure}
  
  \Put(-75, 250){
    \includegraphics[width=.6\textwidth]
    {../../Results/Initial/Figures/RatioAveLegendcompact.pdf}
  }
}

\frame{
  \frametitle{Evidence of Information Effects: Ratio}
  \[
  	y_{it} = \beta_0 + \beta_1 Ratio_{it} + \beta_2 Post2010_t +
  			     \beta_3 (Ratio_{it} * Post2010_t) + \gamma X_{it} + \varepsilon_{it}
  \]
  
  \begin{figure}[htbp]
    \begin{center}
      \includegraphics[width=.5\textwidth]
                      {../../Results/FirstStage/Figures/ratioDinDcompact.pdf}
    \end{center}
  \end{figure}
  \hfill\hyperlink{fr:structural_model}{\beamerbutton{Back}}
}


%
%    RELATIVE VARIBALE IMPORTANCE
%


%
%    MARKET DYNAMICS
%

\frame{
  \frametitle{Market Dynamics - Tuition}
  \label{fr:dynamics}
  \begin{figure}[htbp]
    \begin{center}
      \includegraphics[width=.8\textwidth]
                      {../../Results/SecondStage/Figures/Dynamics/Tuitionquantiles.pdf}
    \end{center}
  \end{figure}
}

\frame{
  \frametitle{Market Dynamics - GPA}
  \begin{figure}[htbp]
    \begin{center}
      \includegraphics[width=.8\textwidth]
                      {../../Results/SecondStage/Figures/Dynamics/UndergraduatemedianGPAquantiles.pdf}
    \end{center}
  \end{figure}
}

\frame{
  \frametitle{Market Dynamics - LSAT}
  \begin{figure}[htbp]
    \begin{center}
      \includegraphics[width=.8\textwidth]
                      {../../Results/SecondStage/Figures/Dynamics/MedianLSATquantiles.pdf}
    \end{center}
  \end{figure}
}

\frame{
  \frametitle{Market Dynamics - Demand}
  \begin{figure}[htbp]
    \begin{center}
      \includegraphics[width=.8\textwidth]
                      {../../Results/SecondStage/Figures/Dynamics/demandquantiles.pdf}
    \end{center}
  \end{figure}
  
  \hfill\hyperlink{fr:first_stage_app_admit}{\beamerbutton{Back}}
}

%
%    VALUE FUNCTIONS
%

\frame{
  \frametitle{Market Dynamics - Tuition}
  \label{fr:val_funs}
  \begin{figure}[htbp]
    \begin{center}
      \includegraphics[width=.6\textwidth]
                      {../../Results/SecondStage/Figures/value.pdf}
    \end{center}
  \end{figure}
  \hfill\hyperlink{fr:conditional_producer_surplus}{\beamerbutton{Back}}
}


\end{document}